\documentclass{article}
\usepackage{amsmath,amssymb,amsfonts,amsthm}
\usepackage{geometry}
\usepackage{caption}
\usepackage{tabularx}
\usepackage{booktabs}
\usepackage{cite}
\usepackage{graphicx}
\usepackage{amsmath}
\usepackage{subfigure}
\usepackage{amssymb}
\usepackage{listings}
\usepackage{xcolor}
\usepackage{float}
\usepackage{graphicx}
\usepackage{subcaption}
\usepackage{algorithm}
\usepackage{algpseudocode}
\usepackage{bm}
\geometry{a4paper, margin=1in}

\title{Mihai Research}
\author{Rongze Wei}
\date{2025.1.31}


\lstset{
    language=Matlab,
    basicstyle=\ttfamily\small,
    keywordstyle=\color{blue},
    commentstyle=\color{gray},
    stringstyle=\color{red},
    showstringspaces=false,
    numbers=left,
    numberstyle=\tiny\color{gray},
    breaklines=true,
}

\begin{document}
\maketitle
\noindent
The equation that we have is $$u_t  = \delta u_{xx} - \delta^{-1}V'(u), V(u) = \frac{1}{4}(1-u^2)^2$$
So, by using the finite difference scheme, the second-order centred differences for partial derivative and the Euler method in time, $$u_{xx} = \frac{u_{j+1}^n - 2u_j^{n} + u_{j-1}^n}{h^2}, u_t = \frac{u_{j}^{n+1} - u_{j}^{n}}{\Delta t}$$
So, the equation becomes that, for $\forall j \in {1,2,3,\cdots,N-1} \quad \text{and} \quad \forall n \in {0,1,\cdots, T-2}$, we have $$\frac{u_j^{n+1} - u_j^n}{\Delta t} = \delta(\frac{u_{j+1}^n - 2u_j^n + u_{j-1}^n}{h^2}) - \frac{1}{\delta}((u_j^n)^3 - u_j^n)$$
which is equivalent to 
\begin{align*}
    u_{j}^{n+1} &= u_j^n + \frac{\delta \Delta t}{h^2}(u_{j+1}^n - 2u_j^n + u_{j-1}^n) - \frac{\Delta t}{\delta}((u_j^n)^3 - u_j^n)\\
    &= (1+ \frac{\Delta t}{\delta}-2\frac{\delta \Delta t}{h^2})u_j^n + \frac{\delta \Delta t}{h^2}u_{j+1}^{n} + \frac{\delta \Delta t}{h^2}u_{j-1}^{n} -  \frac{\Delta t}{\delta}(u_j^n)^3
\end{align*}
Since we have the Dirichlet boundary condiion that$$u(0,t) = u(1,t) = 0, u(x,0) = u_+(x), u(x,T) = u_-(x)$$
so in our numerical simulation, we have $100 \times 100$ points in discretization, so $N = 99, T = 99$. 

\vspace{1em}
\noindent
\textbf{Note that $u_j^n$ denotes the point at the $j$-th position at time $n$.}

\vspace{1em}
\noindent
We only need to consider the case when $j = {1,2,\cdots, 98}$ and $n = {0,1,\cdots, 97}$.

\vspace{1em}
\noindent
Thus, the system of equations that we have is:\\
For $n = 0$,
\[
\renewcommand{\arraystretch}{1.5}
\left\{
\begin{array}{ll}
    u_1^1 &= (1 + \frac{\Delta t}{\delta} - 2\frac{\delta \Delta t}{h^2})u_1^0 + \frac{\delta \Delta t}{h^2}u_2^0 + \frac{\delta \Delta t }{h^2}u_0^0 - \frac{\Delta t }{\delta}(u_1^0)^3 = (1 + \frac{\Delta t}{\delta} - 2\frac{\delta \Delta t}{h^2})u_1^0 + \frac{\delta \Delta t}{h^2}u_2^0 - \frac{\Delta t }{\delta}(u_1^0)^3\\
    u_2^1 &= (1 + \frac{\Delta t}{\delta} - 2\frac{\delta \Delta t}{h^2})u_2^0 + \frac{\delta \Delta t}{h^2}u_3^0 + \frac{\delta \Delta t }{h^2}u_1^0 - \frac{\Delta t }{\delta}(u_2^0)^3\\
    \vdots\\
    u_{98}^1 &= (1 + \frac{\Delta t}{\delta} - 2\frac{\delta \Delta t}{h^2})u_{98}^0 + \frac{\delta \Delta t}{h^2}u_{99}^0 + \frac{\delta \Delta t }{h^2}u_{97}^0 - \frac{\Delta t }{\delta}(u_{98}^0)^3 = (1 + \frac{\Delta t}{\delta} - 2\frac{\delta \Delta t}{h^2})u_{98}^0 + \frac{\delta \Delta t }{h^2}u_{97}^0 - \frac{\Delta t }{\delta}(u_{98}^0)^3
\end{array}
\right.
\]\\
For $n = 1$,
\[
\renewcommand{\arraystretch}{1.5}
\left\{
\begin{array}{ll}
    u_1^2 &= (1 + \frac{\Delta t}{\delta} - 2\frac{\delta \Delta t}{h^2})u_1^1 + \frac{\delta \Delta t}{h^2}u_2^1 + \frac{\delta \Delta t}{h^2}u_0^1 - \frac{\Delta t}{\delta}(u_1^1)^3 = (1 + \frac{\Delta t}{\delta} - 2\frac{\delta \Delta t}{h^2})u_1^1 + \frac{\delta \Delta t}{h^2}u_2^1 - \frac{\Delta t}{\delta}(u_1^1)^3\\
    u_2^2 &= (1 + \frac{\Delta t}{\delta} - 2\frac{\delta \Delta t}{h^2})u_2^1 + \frac{\delta \Delta t}{h^2}u_3^1 + \frac{\delta \Delta t }{h^2}u_1^1 - \frac{\Delta t }{\delta}(u_2^1)^3\\
    \vdots\\
    u_{98}^2 &= (1 + \frac{\Delta t}{\delta} - 2\frac{\delta \Delta t}{h^2})u_{98}^1 + \frac{\delta \Delta t}{h^2}u_{99}^1 + \frac{\delta \Delta t }{h^2}u_{97}^1 - \frac{\Delta t }{\delta}(u_{98}^1)^3 = (1 + \frac{\Delta t}{\delta} - 2\frac{\delta \Delta t}{h^2})u_{98}^1 + \frac{\delta \Delta t }{h^2}u_{97}^1 - \frac{\Delta t }{\delta}(u_{98}^1)^3
\end{array}
\right.
\]\\
For $n = 97$,
\[
\renewcommand{\arraystretch}{1.5}
\left\{
\begin{array}{ll}
    u_1^{98} &= (1 + \frac{\Delta t}{\delta} - 2\frac{\delta \Delta t}{h^2})u_1^{97} + \frac{\delta \Delta t}{h^2}u_2^{97} + \frac{\delta \Delta t}{h^2}u_0^{97} - \frac{\Delta t}{\delta}(u_1^{97})^3 = (1 + \frac{\Delta t}{\delta} - 2\frac{\delta \Delta t}{h^2})u_1^{97} + \frac{\delta \Delta t}{h^2}u_2^{97} - \frac{\Delta t}{\delta}(u_1^{97})^3\\
    u_2^{98} &= (1 + \frac{\Delta t}{\delta} - 2\frac{\delta \Delta t}{h^2})u_2^{97} + \frac{\delta \Delta t}{h^2}u_3^{97} + \frac{\delta \Delta t }{h^2}u_1^{97} - \frac{\Delta t }{\delta}(u_2^{97})^3\\
    \vdots\\
    u_{98}^{98} &= (1 + \frac{\Delta t}{\delta} - 2\frac{\delta \Delta t}{h^2})u_{98}^{97} + \frac{\delta \Delta t}{h^2}u_{99}^{97} + \frac{\delta \Delta t }{h^2}u_{97}^{97} - \frac{\Delta t }{\delta}(u_{98}^{97})^3 = (1 + \frac{\Delta t}{\delta} - 2\frac{\delta \Delta t}{h^2})u_{98}^{97} + \frac{\delta \Delta t }{h^2}u_{97}^{97} - \frac{\Delta t }{\delta}(u_{98}^{97})^3
\end{array}
\right.
\]

\vspace{1em}
\noindent
If we write the iteration in the form of a linear system, we would have,
\[
\renewcommand{\arraystretch}{1.5}
\begin{bmatrix}
  u_1^{n+1} \\
  u_2^{n+1} \\
  u_3^{n+1} \\
  \vdots \\
  \vdots \\
  \vdots \\
  \vdots \\
  u_{98}^{n+1}
\end{bmatrix}
=
\begin{bmatrix}
    1 + \frac{\Delta t}{\delta} - 2\frac{\delta \Delta t}{h^2} & \frac{\delta \Delta t}{h^2} & 0 & 0 & \cdots & 0 \\
    \frac{\delta \Delta t}{h^2} & 1 + \frac{\Delta t}{\delta} - 2\frac{\delta \Delta t}{h^2} & \frac{\delta \Delta t}{h^2} & 0 & \cdots & 0 \\
    0 & \frac{\delta \Delta t}{h^2} & 1 + \frac{\Delta t}{\delta} - 2\frac{\delta \Delta t}{h^2} & \frac{\delta \Delta t}{h^2} & \cdots & 0 \\
    0 & 0 & \ddots & \ddots & \ddots & 0 \\
    0 & 0 & 0 & \frac{\delta \Delta t}{h^2} & 1 + \frac{\Delta t}{\delta} - 2\frac{\delta \Delta t}{h^2} & \frac{\delta \Delta t}{h^2} \\
    0 & 0 & 0 & 0 & \frac{\delta \Delta t}{h^2} & 1 + \frac{\Delta t}{\delta} - 2\frac{\delta \Delta t}{h^2}
\end{bmatrix}
\begin{bmatrix}
u_1^{n} \\
  u_2^{n} \\
  u_3^{n} \\
  \vdots \\
  \vdots \\
  \vdots \\
  \vdots \\
  u_{98}^{n}
\end{bmatrix}
+
\begin{bmatrix}
  (u_1^n)^3 \\
  (u_2^2)^3 \\
  (u_3^n)^3 \\
  \vdots\\
  \vdots\\
  \vdots\\
  \vdots\\
  (u_{98}^n)^3
\end{bmatrix}
\]

\noindent
After discuss with Prof Mihai, we decided to set $$u_{+}(x) = u(x,0) = \text{sin}(\pi x), u_{-}(x) = -\text{sin}(\pi x)$$, and we discretize the space-time domain $[0,1] \times [0,1]$ with sizes $\Delta x = \frac{1}{99}$, $\Delta t = \frac{1}{99}$.\\
To make our life easier, we just try to construct the function $u(x,t)$ on the domain as simple as possible with the boundary condition above. So, we set
\begin{align*}
  u(x,t) &= -t\text{sin}(\pi x) + (1 - t) \text{sin}(\pi x)\\
  &= -2t\text{sin}(\pi x) + \text{sin}(\pi x)\\
  &= (1 - 2t)\text{sin}(\pi x)
\end{align*}

\noindent
Firstly I use Matlab to plot the 3D plot of this function on the domain.
And then I get the matrix $U$, where the last row of U is $u(x,0)$ while the top row is $u(x,1)$.
\begin{lstlisting}
x = linspace(0, 1, 100);
t = linspace(0, 1, 100);

U = zeros(100, 100);

for idx_t = 1:length(t)
    current_t = t(idx_t);
    U(101 - idx_t, :) = (1 - 2 * current_t) * sin(pi * x);
end

[X, T] = meshgrid(x, t);

figure;
surf(X, T, flipud(U));
xlabel('x');
ylabel('t');
zlabel('u(x,t)');
title('u(x,t) = (1 - 2t)sin(\pi x)');
shading interp;
colorbar;
\end{lstlisting}
\noindent
And then I get the matrix $U$, where the row on the top is $u(x,0)$ while the row in the bottom is $u(x,1)$.\\
Next we try to find matrix P where P is the numerical approximation to $$p(x,t) = u_t  -\delta u_{xx} +\delta ^{-1} V'(u)$$
The formula is given by 
\[
P_{i,j+\frac{1}{2}} = \frac{U_{i,j+1} - U_{i,j}}{\Delta t} + \delta^{-1}V'\left(\frac{U_{i,j+1}+ U_{i,j}}{2}\right)- \frac{\delta}{2}\left(\frac{U_{i+1,j+1} - 2U_{i,j+1} + U_{i-1,j+1}}{\Delta x ^2}+ \frac{U_{i+1,j} - 2U_{i,j} + U_{i-1,j}}{\Delta x ^2}\right)
\]
where $i = 1,\cdots,98$,\quad $j = 0,\cdots, 98$,\quad$\delta = 0.05$, $\Delta x = \frac{1}{99}$, $\Delta t = \frac{1}{99}$ and recall that $$V(u) = \frac{1}{4}(1 - u^2)^2, V'(u) = u^3 - u$$

\noindent
After we get $P$, we calculate $S_T(U)$ which is defined as $$S_T(U) = \frac{1}{2}\Delta x\Delta t \sum_{i = 1}^{98} \sum_{j = 0}^{98} P_{i,j+\frac{1}{2}}^2$$
And since the boundary condition, $S_T$ is actually a fucntion of variables $U_{i,j}$ where $i = 1,\cdots, 98$, \quad $j = 1,\cdots, 98$, and the gradient of $S_T$ with respect to each variale is given by 
\[
\frac{\partial S_T}{\partial U_{i,j}} = \left(2\Delta x + \frac{\Delta x \Delta t}{\delta}V''\left(\frac{U_{i,j-1} + U_{i,j}}{2}\right) + \frac{2\delta \Delta t}{\Delta x}\right) P_{i,j-\frac{1}{2}} 
\]
\[
- \left(2\Delta x - \frac{\Delta x \Delta t}{\delta}V''\left(\frac{U_{i,j} + U_{i,j+1}}{2}\right) - \frac{2\delta \Delta t}{\Delta x}\right)P_{i,j+\frac{1}{2}} 
\]
\[
- \frac{\delta \Delta t}{\Delta x}\left(P_{i-1,j-\frac{1}{2}} + P_{i-1,j+\frac{1}{2}} + P_{i+1,j-\frac{1}{2}} + P_{i+1, j+\frac{1}{2}}\right)
\]

\noindent
Well, actually the function $S_T$ is a function of the whole matrix $U$, so there should be $100 \times 100$ random variables, but since the boundary condition, the boundarie are constant so their derivatives are $0$.\\
So the gradient is
\[
\nabla S_T(U) =
\begin{bmatrix} 
   \mathbf{0} \\ 0 \\ \frac{\partial S_T}{\partial U_{1,1}} \\ \vdots \\ \frac{\partial S_T}{\partial U_{98,1}} \\ 0 \\ 0 \\ \frac{\partial S_T}{\partial U_{1,2}} \\ \vdots \\ \frac{\partial S_T}{\partial U_{98,2}} \\ 0 \\ \vdots \\ \\ 0 \\ \frac{\partial S_T}{\partial U_{1,98}} \\ \vdots \\ \frac{\partial S_T}{\partial U_{98,98}} \\ 0 \\ \mathbf{0}
\end{bmatrix}
\]
and the \textbf{BFGS} method is presented as follows,

$U_0$ = U\\
For $k = 0,1,2,...$,\\
$U_{k+1} = U_k + \alpha_k p_k$\\
where the step length $\alpha_k$ is chosen by the wolfe condition
\[
S_T(U_{k+1}) \leq S_T(U_k) + c_1 \alpha_k \nabla S_T(U_k)^T p_k
\]
and 
\[
\nabla S_T(U_{k+1})^T p_k \geq c_2 \nabla S_T(U_k)^T p_k
\]
with $0 < c_1 < c_2 < 1$\\
The descent direction $p_k$ is given by $$p_k = -H_k \nabla f_k$$ where $H_k$ is defined recursively by$$H_{k+1} = (I - \rho_k s_k y_k^T)H_k(I - \rho_k y_k s_k^T) + \rho_k s_k s_k^T$$ where $s_k = U_{k+1} - U_k$, $y_k = \nabla S_T(U_{k+1}) - \nabla S_T(U_k)$, and $\rho_k = \frac{1}{y_k^T s_k}$.\\
For $H_0$, we firstly consider $$H_0 = \gamma_1 I$$ where $\gamma_1 = \frac{y_0^Ts_0}{y_0^Ty_0}$

The algorithm is follows
\begin{algorithm}
  \caption{BFGS Method}
  \begin{algorithmic}[1]
  \State \textbf{Input:} Initial matrix $U_0 = U$, tolerance $\epsilon = 10^{-4}$, $\delta = 0.05$, $\Delta x = \frac{1}{99}$, $\Delta t = \frac{1}{99}$.
  \State \textbf{Initialize:} $H_0 = \gamma_1 I$, where $\gamma_1 = \frac{y_0^T s_0}{y_0^T y_0}$.
  \For{$k = 0,1,2,\ldots$}
      \State Compute descent direction: $p_k = -H_k \nabla S_T(U_k)$.
      \State Choose step length $\alpha_k$ satisfying the Wolfe conditions:
      \begin{align*}
      &S_T(U_{k+1}) \leq S_T(U_k) + c_1 \alpha_k \nabla S_T(U_k)^T p_k, \\
      &\nabla S_T(U_{k+1})^T p_k \geq c_2 \nabla S_T(U_k)^T p_k,
      \end{align*}
      where $0 < c_1 < c_2 < 1$.
      \State Update $U$: $U_{k+1} = U_k + \alpha_k p_k$.
      \State Compute $s_k = U_{k+1} - U_k$ and $y_k = \nabla S_T(U_{k+1}) - \nabla S_T(U_k)$.
      \State Compute $\rho_k = \frac{1}{y_k^T s_k}$.
      \State Update $H_k$ by:
      \[
      H_{k+1} = (I - \rho_k s_k y_k^T)H_k(I - \rho_k y_k s_k^T) + \rho_k s_k s_k^T.
      \]
      \If{$\|\nabla S_T(U_{k+1})\| \leq \epsilon$}
          \State \textbf{Stop} and return $U_{k+1}$.
      \EndIf
  \EndFor
\end{algorithmic}
\end{algorithm}

\noindent
Then, we move to finding the initial boundary condition for $u_+(x,0)$ and $u_-(x,0)$.
These two functions come from the stable equilibrium states of the function$$\mathbb{E}[u] = \frac{1}{2}\int_{0}^{1}(\delta u_x^2 + 2\delta^{-1}\mathbb{V}(u))dx$$
Similarly, we set $q(x) = \delta u_x^2 + 2\delta^{-1}\mathbb{V}(u)$, so the energy function can be written as $$\mathbb{E}[u] = \frac{1}{2}\int_{0}^{1}q(x)dx$$
We use the midpoint rule to compute the temporal integral, and obtaining $$E(U) = \frac{1}{2}\Delta x\sum_{i=1}^{I-1}P_i^2$$ and since here we have $100$ points so $I = 100$, and $P_i$ is the numerical approximation of $q(x_i)$ by central finite difference$$P_i = \delta^{-1}\mathbb{V}(U_i) + \delta (\frac{U_{i+1} - U_{i-1}}{2\Delta x})^2$$
For the boundary condition of U, we have $U_0 = U_{100} = 0$ and the BFGS method requires the gradient, which is given by
\[
\frac{\partial E}{\partial U_1} = \frac{1}{2}\Delta x\left[\delta^{-1}\frac{1}{2}(1-U_1^2)(-2U_1) + 2\delta(\frac{U_3 - U_1}{2\Delta x})(-\frac{1}{2\Delta x})\right]
\]
\[
\frac{\partial E}{\partial U_2} = \frac{1}{2}\Delta x\left[2\delta\frac{U_2}{2\Delta x}(\frac{1}{2\Delta x})+\delta^{-1}\frac{1}{2}(1 - U_2^2)(-2U_2) + 2\delta(\frac{U_4 - U_2}{2\Delta x})(-\frac{1}{2\Delta x})\right]
\]
\[
\frac{\partial E}{\partial U_i} = \frac{1}{2}\Delta x\left[2\delta(\frac{U_i - U_{i-2}}{2\Delta x})(\frac{1}{2\Delta x})+\delta^{-1}\frac{1}{2}(1 - U_i^2)(-2U_i) + 2\delta(\frac{U_{i+2} - U_i}{2\Delta x})(-\frac{1}{2\Delta x})\right], \quad \text{for} \quad i = 3,\cdots,97
\]
\[
\frac{\partial E}{\partial U_{98}} = \frac{1}{2}\Delta x\left[2\delta(\frac{U_{98} - U_{96}}{2\Delta x})(\frac{1}{2\Delta x}) + \delta^{-1}\frac{1}{2}(1 - U_{98}^2)(-2U_{98}) + 2\delta\frac{-U_{98}}{2\Delta x}(-\frac{1}{2\Delta x})\right]
\]
\[
\frac{\partial E}{\partial U_{99}} = \frac{1}{2}\Delta x\left[2\delta(\frac{U_{99} - U_{97}}{2\Delta x})(\frac{1}{2\Delta x}) + \delta^{-1}\frac{1}{2}(1 - U_{99}^2)(-2U_{99})\right]
\]

\noindent
And the BFGS method present as follow.
$\textbf{U}_0 = \text{sin}(\pi x), H_0 = I$\\
The BFGS method defines the next one as $$\textbf{U}_{k+1} =\textbf{U}_k + \alpha_k p_k$$, where $\alpha_k$ is chosen to satisfy the wolfe conditions
\[
E(U_{k+1}) \leq E(\textbf{U}_k) + c_1 \alpha_k \nabla E(\textbf{U}_k)^T p_k
\]
and 
\[
\nabla E(\textbf{U}_{k+1})^T p_k \geq c_2 \nabla E(\textbf{U}_k)^T p_k
\]
with $0 < c_1 < c_2 < 1$\\
The search direction $p_k$ is given by $$p_k = -H_k\nabla E_k$$
where $H_k$ is defined recursively by $$H_{k+1} = (I -\rho_k s_k y_k^T)H_k(I - \rho_k y_k s_k^T)+\rho_ks_ks_k^T$$, where $$s_k = E(\textbf{U}_{k+1}) - E(\textbf{U}_k),\quad y_k = \nabla E_{k+1} - \nabla E_k, \quad \rho_k = \frac{1}{y_k^Ts_k}$$
and finally the iteration would stop if the norm of the gradient is smaller than $\epsilon = 10^{-4}$.

\vspace{1em}
\noindent
Unfortunately, we saw strong oscillation in the figure output and so 
\end{document}